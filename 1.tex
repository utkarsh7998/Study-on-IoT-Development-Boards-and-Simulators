\documentclass[a4paper,11pt]{article}
\usepackage{iotsubmit}
\usepackage{longtable}
\usepackage{tabularx}
\usepackage{url}
\def\UrlBreaks{\do\/\do-}
\begin{document}

\initiotsubmision{1}                    % assignment number
{Kajal Sethi, Utkarsh Srivastava}                            % group member names
{kajals21@iitk.ac.in, utkarshs21@iitk.ac.in}    % email addresses for all members
{21111033, 21111063}             % roll numbers for all members (according to names)

\begin{iotsolution}
A microprocessor development board is a printed circuit board containing a microprocessor and the minimal support logic needed by a person that wants to become acquainted with the microprocessor on the board and to learn to program it. It also serves users of the microprocessor as a method to prototype applications in products \cite{[B0]}.

IoT boards are also known development boards or prototype boards. Each board consists of a circuit printed on it. The circuit is designed in such a way that it executes desired operations. Here are the features of some selected IoT development boards.
\begin{enumerate}
    \item \textbf{Omega2:} The Omega2 modules are Onion’s Linux-based Wi-Fi development boards. Equipped with a MT7688 SoC that features a 580 MHz CPU cycles. It has 128 MB RAM and 32 GB of Flash. It supports 2.4 GHz IEEE 802.11 b/g/n Wi-Fi and 10M/100M wired ethernet network connectivity and operates at 3.3V \cite{[A7]}. It features 32 pins at a 2mm pitch, up to 18 pins can be used as user controllable GPIOs. The board also features Wi-Fi radio, USB drive, MicroSD, UARTs, I2C and SPI. One of it's limitation is that, it lack of an analog input and it can be solved with an extension board or an I2C converter \cite{[B1]}.
    \item \textbf{ARGON:} Particle.io is powerful Wi-Fi enabled development kit that can act as a standalone Wi-Fi endpoint. It is based on the Nordic nRF52840 SoC operating at 64MHz and need a 3.3VDC SMPS power supply \cite{[B2]}. It has Espressif ESP32-D0WD coprocessor(4MB flash for ESP32) and supports BLE and Bluetooth. It has photodiode, 20 mixed signal GPIO (6 x Analog, 8 x PWM) with UART, I2C and SPI interface and Micro USB, Li-Po charging and battery connector, JTAG (SWD) Connector, on-board 2.4GHz PCB antenna, two U.FL connectors for external antennas. The board supports DSP instructions and FPU calculations, ARM TrustZone CryptoCell-310 Cryptographic and has a security module \cite{[B3]}. 
    \item\textbf{ESP32:} Produced by Espressif, ESP32 comes with integrated Wi-Fi, BLE and dual-mode Bluetooth and a memory of 320 KiB RAM and 448 KiB ROM. This board employs Xtensa LX7 dual-core microprocessor (operating at 160 or 240 MHz and performing at up to 600 DMIPS) or a single-core RISC-V microprocessor. It has built-in antenna switches, RF balun, power amplifier, low-noise receive amplifier, filters, and power-management modules \cite{[B4]}. The board is equipped with 34 programmable GPIOs, 12-bit SAR ADC, touch sensors, SPI, I2S, UART and I2C interfaces. This board supports Ethernet MAC interface, CAN bus, Infrared remote controller, LED PWM, Hall effect sensor, Ultra low power analog pre-amplifier, Cryptographic hardware acceleration, Wake up from GPIO interrupt, timer, ADC measurements and capacitive touch sensor interrupt \cite{[A6]}.
    \item\textbf{Teensy 4.1:} Teensy 4.1 is 32-bit ARM Cortex M7 with a clock running at 600MHz and the memory consisting of 8MB Flash, 1MB RAM and 4KB EEPROM . It supports 10/100 Mbit Ethernet. It has 40 GPIOs (18 analog inputs) and comes with I2C, SPI, UART and CAN communication support. The board also features USB drive, 35 PWM output pins,18 analog input pins, serial ports, SPI and I2C ports, I2S/TDM and S/PDIF digital audio port, CAN Bus, SDIO (4 bit) native SD Card port, 32 general purpose DMA channels, Cryptographic Acceleration, RTC, Programmable FlexIO, Pixel Processing Pipeline and Peripheral cross triggering. The sensor associated with this board is temperature. The board operates at a power of 3.3V. It does not have Wi-Fi or BLE communication \cite{[B5]}.
    \item\textbf{Raspberry Pi 4:} Equipped with Broadcom BCM2711, Quad center Cortex-A72 (ARM v8) 64-bit SoC processor Raspberry Pi is a universally useful device. It has 2GB RAM in the default version. For communication, Wi-Fi, BLE, Bluetooth and Ethernet is available. It has 40 GPIOs, with SPI, I2C, and UART interfaces, USB port, HDMI port, DSI display, and CSI camera, 4-pole stereo audio and composite video port, OpenGL, Micro-SD card and temperature sensor. Various OS that can be used with Raspberry Pi are Raspbian, OSMC, Ubuntu core, Kali Linux, Windows IoT, Debian, etc. It operates at a relatively high voltage of 5V and lacks an analog input \cite{[B6]}.
\end{enumerate}
\subsubsection*{Table of Comparison}
\begin{center}
\label{tab:title}
\begin{longtable}{|p{2cm}| p{2.2cm} | p{2.5cm} | p{2.5cm} | p{2.2cm} | p{2cm} | }
\hline 
\textbf{Device Name} &  \textbf{Omega2} & \textbf{ARGON} & \textbf{ESP32} & \textbf{Teensy 4.1} & \textbf{Raspberry Pi 4} \\
\hline
\hline
\textbf{Processor} & MT7688 SoC processor &  Nordic nRF52840 and Espressif ESP32 processors & Xtensa LX7 dual-core processor or a single-core RISC-V processor &  32-bit ARM Cortex M7 processor & Broadcom BCM2711, Quad center Cortex-A72 (ARM v8) 64-bit SoC processor  \\
\hline
 \textbf{Company} & Onion.io & Particle.io & Espressif Systems & PJRC & Raspberry Pi \\
 \hline
 \textbf{Clock Speed} & 580 MHz & 64 MHz and ESp32 - 2.4 GHz  &  240 MHz & 600 MHz &  1.5 GHz \\
 \hline
\textbf{Storage} & 128 MB of RAM and 32 GB of Flash &	1MB flash, 256KB RAM and 4MB flash for ESP32&	320 KiB RAM, 448 KiB ROM&	8MB Flash, 1MB RAM and 4KB EEPROM&	2GB RAM   \\
\hline
\textbf{Cost} & \$21.70 &	\$36.19&	\$11.95& \$26.85&	\$45.02\\
 \hline
\textbf{Operating Voltage}  & 3.3V & 3.3V DC & 3.3 V DC &3.3V & 5V \\
\hline
\textbf{Sensors} & Temperature & Photodiode&	Hall effect and touch, LED&	Temperature&	 Temperature\\
 \hline
\textbf{OS Support} & Linux & Device OS & Mongoose OS & Windows, linux, Macintosh&	Raspbian, Debian, OSMC, Windows IoT, Ubuntu core, Kali Linux\\
\hline
 \textbf{Supported communication protocols} & Wi-Fi, radio, Ethernet, I2C bus, UART and 2 PWM&	BLE and Bluetooth&	Wi-Fi, dual mode Bluetooth, BLE,  I2C, SPI and UART&	Ethernet, I2C, SPI, UART and CAN&	WiFi, Bluetooth, BLE, Ethernet, SPI, I2C, and UART \\
 \hline
 \textbf{GPIO} & 18 & 20 mixed signal & 34 programmable  & 40 & 40\\
\hline
\textbf{Limits and Drawbacks} & Lack of an analog input(solved by extension board).&	Doesnot support Wi-Fi& &		Does not have WiFi or BLE communication&	Lacks a analog input and realatively high operating voltage\\
\hline
\caption{Comparative Study of IoT Boards  \cite{[A7]} \cite{tewari2021comparative}}
\end{longtable}

\end{center}
\subsubsection*{Summary}
The different features that go with each IoT contraption stage makes the board ideal for explicit applications. And therefore while picking a development board for specific Iot application, one could consider the above device features.

As it has a lot of spare space and a great processor, the Raspberry Pi is suitable for a worker-based IoT application. Furthermore, it supports a variety of programming languages, such as Node.js, that can be used to create worker-side apps. The Raspberry Pi is capable of connecting to both LAN and Wi-Fi networks.\cite{tewari2021comparative}

The ESP32 has a very simple equipment core configuration and is best suited for customer applications such as data logging and actuator control from online worker apps. It has Bluetooth (v4.2 and v5.0) and WiFi integrated, so you don’t need any other module to start communicate with the world. Engineered for wearable and mobile devices, it has low-power consumption and is capable of working in a wide range of temperature.

Particle Argon is a useful IoT prototyping platform that allows for remote programming, simple code migration, and rapid venture development. It has Low Power Consumption, Low-Cost Hardware, High Network Reliability, and Local Communications to ensure that messages are delivered quickly and reliably.\cite{tewari2021comparative}

The Omega's tiny form factor (which makes it easier to integrate into any design), low power draw, processing and networking capabilities, and Linux flexibility make it perfect for the type of linked and intelligent applications associated with IoT. The board can be used for designing devices intended for headless computers with no graphical interfaces in Embedded systems.

%\subsubsection*{References}
%here we list all the sources used.

\end{iotsolution}

\begin{iotsolution}


A simulator should always be considered at an early phase of design, development, and testing of IoT applications and devices without setting up actual IoT boards as it saves us from unnecessary hardware changes as the requirement changes. Simulators give a higher level of abstraction and help to test futuristic IoT devices. The major advantage of using an IoT Simulator is that it helps to test potential use cases and detect faults at the early stage. Moreover, it also helps to analyze the response of different parts of IoT system in a virtual environment before setting it up physically.


\subsubsection*{Table of Comparison}
\begin{center}
 \label{tab:title2}
\begin{longtable}{|p{2.5cm} |p{2.5cm} |p{2cm} |p{2cm} |p{2cm} |p{2cm} |}
\hline
\textbf{Simulator} &  NCTUns 6.0 & NS-Series&IoTIFY & Bevywise-IoT & Ansys-IoT \\
\hline
\textbf{Scope} & Sensor Networks & Network & Hardware Connection & IoT Device & IoT Industry\\
\hline
\textbf{Type}  & Discrete-event & Discrete-event & Mobile App & Broker & Autonomous\\
\hline
\textbf{Programming Language} & C++ & C++, C Sharp & Python, Java & Python, Java & Python, Java\\
\hline
\textbf{IoT Architecture Layers} &Network Data Link &Perceptual Network & Application Network
 & Network & Network \\
 \hline
\textbf{Scale of Operation} & Large Scale & Large Scale & Large Scale & Large Scale & Large Scale	\\
\hline
\textbf{Built-in IoT Standards} & 802.11p WiMAX MANETS Optical Network
 & 802.15.4 LoRaWAN & Real Time & Real Time & Real Time \\
 \hline
\textbf{API Integration} & SOAP & REST & REST & REST & REST\\
\hline
\textbf{Cyber Resilience Simulation} & Yes & No & Yes & No & Yes \\
\hline
\textbf{Service Domain} & Open Source &Generic &Smart City &Smart City & Industry\\
\hline
\textbf{Security Measures} & High & High & High & Medium & High\\
\hline
\caption{Comparison of Selected IoT Simulators \cite{Simulators}}
\end{longtable}

\end{center}

\subsubsection*{Summary}

  The simulators listed in Table 2 are explained below.
\begin{enumerate}
    \item \textbf{NCTUns 6.0}: Released on November 1, 2002, NCTUns is an open-source software running on Linux that provides an easy-to-use integrated GUI environment for users to conduct simulation. It is a powerful and valuable tool, used for network research, planning, and education. It is mainly used for wireless vehicular network research. \cite{NCTUns}.
    \item \textbf{NS-Series}: They are a series of simulators which include NS-1, NS-2, NS-3 and NS-4. Particularly, NS-3 is used in internet systems, wireless networks and more. It is also one of the fastest and the most memory efficient simulator. However, it does not have GUI to build topology and its visualizations are still experimental. \cite{Simulators}
    \item \textbf{IoTIFY}: IoTIFY is a software extension surroundings for IoT besides embedded hardware territories. By resorting to the device virtualization and intelligent device simulator, it affords a digital lab for building analytical modeling, embedded prototypes, load testing solution, and a com- munity simulation for analytical scaling, virtual devices in the fog environment, and records generation. \cite{Simulators}
    \item \textbf{Bevywise-IoT}: Bevywise IoT Simulator is an intelligible simulation tool to simulate tens of thousands of MQTT clients in a single box. It provides an end to end product and solution for IoT \& Industrial-IoT. It can be used to demo, develop, and test in a real-time IoT environment. \cite{Simulators}
    \item \textbf{Ansys-IoT}: ANSYS-IoT simulator can help to the various challenges of the IoT devices. It can help to discover how to validate and even improve the reliability, power consumption, longevity, and overall integrity of smart sensors in IoT devices. \cite{Simulators}
\end{enumerate}

%

\end{iotsolution}

\begin{iotsolution}
IoT development boards are used for developing a wide variety of applications ranging from temperature sensors to automated car parking systems. So it is important to choose a correct platform for IoT projects to build an application.

\subsubsection*{Onion Omega2}
Omega2 is powered by a full processor which supports many programming languages and many simultaneous processes. The Omega2 modules are powerful and flexible enough to be the central module of any IoT device. And therefore as an IoT computer, the Omega2 really shines when used for IoT projects. There is no need to reinvent the wheel as file systems, I/O control, and package management are already taken care of.\cite{[A1]} It also has a file system with storage. The combination of the Omega’s small form factor makes it easy to embed in any design also it is low powered board with processing and networking capabilities, and the flexibility that comes from running Linux make it ideal for the type of connected and intelligent applications associated with IoT.

Some of the applications of the board are \cite{[A1]}:
\begin{enumerate}
    \item \textbf{Weather Station}:  Reports the collected temperature and humidity data to IBM’s Watson IoT Platform

    \item \textbf{Time-Lapse Camera}

    \item \textbf{Alarms based on an Online Calendar }

\item \textbf{Thermal Printer}

\item \textbf{Smart Plant}
    \begin{enumerate}
    \item \textbf{Measure Plant Data} - Add smarts to your plant by measuring its soil moisture level.
    
    \item \textbf{Visualizing Plant Data} - Send plant data to the Losant IoT Platform and check in on your plant from anywhere by looking at the nicely visualized data.
    
    \item \textbf{Twitter Alerts} - Update the Losant workflow to notify you with a Tweet when your plant needs watering.
    
    \item \textbf{Automatic Plant Watering} - Add a water pump to your setup and update the Losant workflow to automatically water your plant when it needs watering.
    
    \item \textbf{A Single Power Supply} - Update the smart plant setup so the Omega and pump can be powered with a single supply.
\end{enumerate}

\item \textbf{Temperature-Based Smart Fan.}

\item \textbf{IoT Lock.}
\end{enumerate}

\subsubsection*{Raspberry Pi 4}
Raspberry-pi is basically credit card sized, small computer which is capable of interacting with outside world and can be easily accessible to learn how to program in languages like Python. It connects to internet through various different languages like Python, Java, Javascript and supports various operating system like an Linux, Raspbian, Debian, OSMC, Windows IoT, etc.\cite{[A3]}.

For developing an IoT application, Internet access is a prime necessity and Raspberry-Pi can eliminate the complex interfacing like GSM/GPRS also, since Raspberry Pi comes with an Linux operating system it is fully functional whereas Arduino uno is just a part of computer. It can run multiple programs and is a multitasking\cite{[A4]}\cite{[A5]}.

Following are some of the applications of Raspberry Pi\cite{[A2]}:
\begin{enumerate}
\item \textbf{Music machines.}

\item \textbf{Parent detectors.}

\item \textbf{Weather stations.}

\item \textbf{Tweeting birdhouses with infra-red cameras.}

\item \textbf{Teachable Machine.}

\item \textbf{Smart Mirror Touchscreen}: with Face Recognition.

\item \textbf{Smart Home IoT System Based on Raspberry Pi 4.}

\item \textbf{Air Quality Monitor using Raspberry Pi 4, SPS30 and Azure.}

\item \textbf{Cable-Mounted Robot for Near Shore Monitoring }\cite{snow2020design}
\item \textbf{Offline Speech Recognition on Raspberry Pi 4 with Respeaker.}

\item \textbf{Safety Measures (Social Distancing)}: For detecting the places with most interaction and detecting the path of the workers. 

\item \textbf{WebCam Motion Detection With Motioneyeos Using Raspberry Pi.} 

\item \textbf{Wireless Video Surveillance Robot using Raspberry Pi.} 
\end{enumerate}

\subsubsection*{ESP32}
The ESP32 has a very simple equipment core configuration and is best suited for customer applications such as data logging and actuator control from online worker apps\cite{[A7]}. It has Bluetooth (v4.2 and v5.0) and WiFi integrated, so you don’t need any other module to start communicate with the world. Engineered for wearable and mobile devices, it has low-power consumption and is capable of working in a wide range of temperature.\cite{[A6]}

Following are few applications of the board:
\begin{enumerate}
\item \textbf{Internet of Things Based Home Monitoring and Device Control Using Esp32 }\cite{pravalika2019internet}

\item \textbf{ESP32 Based Smart Surveillance System }\cite{rai2019esp32}

\item \textbf{Using the ESP32 Microcontroller for Data Processing} \cite{babiuch2019using}

\item \textbf{Design and implementation of a low cost web server using ESP32 for real-time photovoltaic system monitoring }\cite{allafi2017design}

\item \textbf{Solar Water Pumping System Control Using a Low Cost ESP32 Microcontroller }\cite{biswas2018solar}
\end{enumerate}

\subsubsection*{Teensy 4.1}
The following are the application of Teensy 4.1\cite{[A8]}:
\begin{enumerate}
\item \textbf{Convolution Software Defined Radio.} 

\item \textbf{Dancing Fountain.} 

\item \textbf{ SimpleRick}: Low Cost Ultrasound Imaging.

\item \textbf{SEGA ROM Reader.} 

\item \textbf{SimpleSynth}: Musician and maker Ghost in Translation has made a simple polyphonic FM synth called SimpleSynth.
\item \textbf{Cable-Mounted Robot for Near Shore Monitoring  }\cite{snow2020design1}
\item \textbf{UNISON: a Novel System for Ultra-Low Latency Audio Streaming Over the Internet }\cite{werner2021unison}
\end{enumerate}

\subsubsection*{Argon Particle.io}
ARGON is a powerful iot development board which is WiFi-enabled. The major advantage of using this board is that it is same as Photon, with added feature of Bluetooth which makes it more desirable. Argon is an extraordinary iot development board for interfacing projects or as a gateway to connect a group containing endpoints.
Following are few applications of the  board:

1.  Smart Desk 

2.  House Plant Watering System.

3.  Urban noise and air pollution monitoring system.

4.  Sump Pump Monitor.

5.  Water Bank Powered by IoT.

6. FireFicator.

\subsubsection*{}


\end{iotsolution}

\bibliographystyle{abbrvnat}
% \bibliographystyle{abbrv}
\bibliography{references.bib}

\end{document}

